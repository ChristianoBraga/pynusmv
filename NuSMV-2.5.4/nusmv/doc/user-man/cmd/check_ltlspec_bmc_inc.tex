% -*-latex-*-
\begin{nusmvCommand} {check\_ltlspec\_bmc\_inc} {Checks the given LTL
specification, or all LTL specifications if no formula is given, 
using an incremental algorithm.
Checking parameters are the maximum length and the loopback value}

\cmdLine{check\_ltlspec\_bmc\_inc [-h ] | [-n idx | -p "formula [IN
      context]" | -P "name" ] [-k max\_length] [-l loopback]}

For each problem this command incrementally generates many
satisfiability subproblems and calls the SAT solver on each one of them. The
incremental algorithm exploits the fact that subproblems have common
subparts, so information obtained during a previous call to the SAT solver
can be used in the consecutive ones. Logically, this command does
the same thing as \code{check\_ltlspec\_bmc} (see the description on
page \pageref{checkLtlspecBmcCoomand}) but usually runs
considerably quicker. A SAT solver with an incremental interface
is required by this command, therefore if no such SAT solver is
provided then this command will be unavailable.

See variable \varName{use\_coi\_size\_sorting} for changing properties
verification order.

\begin{cmdOpt}

\opt{-n \parameter{\natnum{\it index}}}{\natnum{\it index} is the numeric index 
of a valid LTL specification formula actually located in the properties database.}
       
\opt{-p \parameter{"\anyexpr [IN context]"}}{Checks the \anyexpr specified on
the command-line. \code{context} is the module instance name which
the variables in \anyexpr must be evaluated in.}
            
\opt{-P \parameter{\natnum{name}}}{Checks the LTL property named \natnum{name} in
the property database.} 

\opt{-k \parameter{\natnum{\it max\_length}}}{\natnum{\it max\_length} is the maximum problem
bound must be reached. Only natural numbers are valid values for this
option. If no value is given the environment variable \envvar{\it
bmc\_length} is considered instead.}

\opt{-l \parameter{\set{\it loopback}{\range{0}{max\_length-1},
       \range{-1}{bmc\_length}, X, *}}}{The {\it loopback} value may be: }
       \tabItem{a natural number in (0, {\it max\_length-1}). A positive sign (`+')
       can be also used as prefix of the number. Any invalid
       combination of length and loopback will be skipped during the
       generation/solving process.}
       \tabItem{a negative number in
       (-1, -{\it bmc\_length}). In this case {\it loopback} is
       considered a value relative to {\it max\_length}.  Any invalid
       combination of length and loopback will be skipped during the
       generation/solving process.}
       \tabItem{the symbol
       `\varvalue{X}', which means ``no loopback".}
       \tabItem{the
       symbol `\varvalue{*}', which means ``all possible loopback from
       zero to {\it length-1}" .}

\end{cmdOpt}
\end{nusmvCommand}
